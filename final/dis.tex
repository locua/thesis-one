% Created 2020-05-01 Fri 19:02
% Intended LaTeX compiler: pdflatex
\documentclass[12pt]{report}
\usepackage[utf8]{inputenc}
\usepackage[T1]{fontenc}
\usepackage{graphicx}
\usepackage{grffile}
\usepackage{longtable}
\usepackage{wrapfig}
\usepackage{rotating}
\usepackage[normalem]{ulem}
\usepackage{amsmath}
\usepackage{textcomp}
\usepackage{amssymb}
\usepackage{capt-of}
\usepackage{hyperref}
\setlength{\parindent}{0pt}
\usepackage[margin=1in]{geometry}
\usepackage{emptypage}
\usepackage{tikz}
\usetikzlibrary{graphs,graphs.standard,bayesnet,arrows.meta,shapes.arrows}

% Colorlinks sets pdfborder to zero.
%\hypersetup{colorlinks,linkcolor=,urlcolor=gray}

%\usepackage[a4paper,top=2.25cm,bottom=2.25cm,left=3.0cm,right=3.0cm]{geometry}

\makeatletter
\renewenvironment{abstract}{%
    \if@twocolumn
      \section*{\abstractname}%
    \else %% <- here I've removed \small
      \begin{center}%
        {\bfseries \Huge\abstractname\vspace{\z@}}%  %% <- here I've added \Large
      \end{center}%
      \quotation
    \fi}
    {\if@twocolumn\else\endquotation\fi}
\makeatother

%\pagecolor{yellow!10}

\documentclass{book}
\usepackage{sectsty}
\chapternumberfont{\large} 
\chaptertitlefont{\huge}

\author{Louis James}
\date{\today}
\title{A system for exploring haptic and spatial interaction}
\hypersetup{
 pdfauthor={Louis James},
 pdftitle={A system for exploring haptic and spatial interaction},
 pdfkeywords={},
 pdfsubject={Final year project for Creative Computing},
 pdfcreator={Emacs 26.3 (Org mode 9.3.6)}, 
 pdflang={English}}
\begin{document}

\maketitle


\renewcommand{\abstractname}{Acknowledgements}
\begin{abstract}
 Thanks to my family, Florent, Chudleigh dwellers, Jamie ...
\end{abstract}
\newpage
\renewcommand{\abstractname}{Abstract}
\begin{abstract}
This project\ldots{}
\end{abstract}
\tableofcontents
\listoffigures
\chapter{Introduction}
\label{sec:orgea35e9f}
\chapter{Background}
\label{sec:org96f72a8}

The motivation for this project stems in part from a feeling of frustration in
how working on computers can often be a constricted affair and a pondering over
how we might expand the \emph{keyboard-mouse-monitor} model to improve the utility of
computers regarding our own perceptive abilities. How might a spatial and more
\emph{haptic} environment for interaction create an improved space for thinking with
computers as well as our physical health.

\section{Definitions}
\label{sec:org2af3ec0}
\begin{enumerate}
\item Computing
\label{sec:org3d867a2}
\item keyboard-mouse-monitor model (kmm model) (?)
\label{sec:org45210ef}
\item Cognition
\label{sec:org214fa66}
\item Exocortex
\label{sec:org07f9845}
\end{enumerate}

\section{Beginning with the Exocortex}
\label{sec:org33257ea}

I started off looking at \emph{Exocortexes} and other personal archiving systems.
Systems that allow the user to externalise thought and memory. This could be via
simply storing and organising work and ideas efficiently and methodically or
unifying many tasks or different workflows into a singular interface. \\

Org mode is a good example of such a system. Org mode is a "computing
environment for authoring mixed natural and computer language documents"
\cite{Schulte:Davison:Dye:Dominik:2011:JSSOBK:v46i03}. It is designed for taking
notes, producing documents and organising and runs inside of the text editor,
Emacs. It has the ability to export to different formats such as HTML, \LaTeX{} and
supports "outlining, note-taking, hyperlinks, spreadsheets, TODO lists, project
planning, GTD" as well as literate programming, all in plain-text
\cite{Schulte:Davison:Dye:Dominik:2011:JSSOBK:v46i03}. (it is incidentally what
this document is produced with) \\


Another point of reference when I was looking at externalised 'artificial
information-processing systems' was Devine Lu Linvega's Exocortex and personal
logging system \href{https://wiki.xxiivv.com/site/nataniev.html}{XXIIV -- nataniev}. \emph{XXIIV} is a personal archive and log with
documentation of Linvega's personal tools and artworks. The site has gone
through some changes since I first came across it. Originally a static,
javascript and lisp based website with diaries, blog type posts and categorised
personal logs. It is now somewhat stripped back in style and has been rewritten
in \href{https://en.wikipedia.org/wiki/C99}{C (C99)} which generates the HTML pages. The work contains a selection of
esoteric programming tools including a synthetic language, games, software all
logged using their own \emph{Arvelie calendar} \cite{DevineNataniev}. \\

Both these two systems have their own specific use-cases, \emph{Org-mode} in academia
and science and \emph{XXIIV}, a personal archive and experimental system. They both
are exist in the contemporary and prevailing \emph{keyboard-mouse-monitor} paradigm
of computer interaction but push the boundaries of cognition in this medium,
particularly regarding memory and productivity.



\section{Dynamicland}
\label{sec:org64c773d}

One of my original points of reference was \emph{Dynamicland}, a research project in
Oakland, USA. The aim of the project is to implement a new more powerful and
accessible model of computing.

\begin{quote}


In Oakland, we built the first full-scale realization of the vision, inviting
thousands of people into our space to collaborate. Together, these artists,
scientists, teachers, students, programmers, and non-programmers created
hundreds of projects that would have been impossible anywhere else.
-- Dynamicland.org 
\end{quote}


\emph{Dynamicland} is a communal computer where the building is the computer (ENIAC).
Programs are embodied in the room on pieces of colour-coded paper. The programs
are recognised via the codes and their code, stored in a database is then run,
it can also \emph{read} code using OCR but generally the code is there \href{https://thenewstack.io/dynamicland-rethinks-computer-interfaces/}{symbolically}.
Projectors on the ceiling transform the paper and workbenches into whatever the
programmer decides. This relatively simple model makes for an exciting new
ecosystem for collaborative computing and expressive programming. Victors,
highlights his ideas for the progression of computing and interaction in a
series of talks (available online) and on his \href{http://worrydream.com}{website}. In his talk "Seeing
Spaces" he talks of a new kind of maker-space which allow makers to see across
time and possibilities. \emph{Dynamicland} offers a computational medium which allows
for full use of the human senses and a more \href{https://vimeo.com/115154289}{humane representation of thought}
\cite{VictorKayDynamicLand}. \\

\begin{figure}[htbp]
\centering
\includegraphics[width=12cm]{assets/realtalk-os.jpg}
\caption{RealtalkOS, the operating system of Dynamicland}
\end{figure}  


\emph{DL} was the inspiration for the main physical and technical model for
this project, an \emph{augmented} workspace either on the floor or a table which is
projected onto. A camera/s pointing down onto the projection space is the sensor
for detecting interaction, with the projector as the actuator. This base model can be
seen in Figures \ref{pp-schema} and  \ref{systemSchema}.



\section{Paper programs - open source}
\label{sec:orgedceb46}

Looking to find some of the code for \emph{Dynamicland} and a more detailed
specification was available online I stumbled across \emph{Paper Programs}
(\emph{Dynamicland} has an 'open-source model', but it is only open if you can visit
it physically as the source code is physically in the space). \emph{Paper Programs}
is a browser-based partial clone of \emph{Dynamicland}. This was another starting
point for playing around with but I found that I couldn't set it up and have it
stable enough to develop on. It also suffers from being quite slow, due to the
Computer Vision and graphics being done in the browser (it uses a
version of OpenCv compiled to \href{https://webassembly.org/}{WebAssembly}) \cite{JpPaperPrograms}.


\section{Sage digital research}
\label{sec:org5258e62}

Ethnomethodology

Embodied Cognition

Haptic interfaces

\section{Nielsen: augmenting ltm and using ai to augment human-i ??????}
\label{sec:org6956ad0}

\chapter{Specification and context}
\label{sec:org37c65bc}

\begin{figure}[htbp]
\centering
\includegraphics[width=15cm]{assets/pp-diag.png}
\caption{The initial physical schema: \emph{Paperprograms} \label{pp-schema}}
\end{figure}


\chapter{Project in depth}
\label{sec:orgc7b38b2}

See system schema Fig.  

\begin{figure}[htbp]
\centering
\includegraphics[width=15cm]{assets/project-schema-final.png}
\caption{System schema \label{systemSchema}}
\end{figure}


\chapter{Creative process}
\label{sec:org1c7f2b4}
\chapter{Debugging and problem solving}
\label{sec:org7063455}
\chapter{Evaluation and Conclusions}
\label{sec:org40f93e2}
\bibliographystyle{ieeetr} 
\bibliography{references}
\end{document}
